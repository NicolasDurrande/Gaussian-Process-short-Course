\documentclass[12pt]{scrartcl}

\usepackage[utf8]{inputenc}

\usepackage{mathpazo} % math & rm
% \linespread{1.05}        % Palatino needs more leading (space between lines)
\usepackage[scaled]{helvet} % ss
\usepackage{courier} % tt
\normalfont
\usepackage[T1]{fontenc}

\usepackage{amsthm,amssymb,amsbsy,amsmath,amsfonts,amssymb,amscd}
\usepackage{dsfont}
\usepackage{tasks}
\usepackage{enumitem}
\usepackage[top=2cm, bottom=3cm, left=3cm , right=3cm]{geometry}
\usepackage{tikz}
\usepackage[hidelinks]{hyperref}

\usetikzlibrary{automata,arrows,positioning,calc}

\begin{document}
\begin{center}
	\rule{\textwidth}{1pt}
	\\ \ \\
	{\LARGE \textbf{Lab 4 -- Efficient global optimization}}\\
	\vspace{3mm}
	{\large Short course on Statistical modelling for optimization\\ \vspace{3mm}}
	{\normalsize N. Durrande - J.C. Croix, Universidad Tecnol\'ogica de Pereira, 2017}\\
	\vspace{3mm}
	\rule{\textwidth}{1pt}
	\vspace{5mm}
\end{center}
The aim of this lab session is to obtain the parameter settings that provides the longest catapult shots.

%%%%%%%%%%%%%%%%%%%%%%%%%%%%%%%%%%%%%%%%%%%%%%%%%
% \subsection*{Data}
% Some data from one of the group is provided, as well as one GPR model. If you are not confidemt in your data or in your model, feel free to use them.

% \subsection*{Questions}

\paragraph{Q0.} First of all, you should make sure you are happy with the model you have and that you have validated both the predicted mean and variance.

\paragraph{Q1.} Before starting the minimization procedure, there is one important thing you should do when you build your model... what is it?

\paragraph{Q2.} A function \texttt{EI} returning the expected improvement is provided in the script. However, this function is only valid for data that is not noisy. Modify it accordingly using one of the two methods discussed during the lecture.

\paragraph{Q3.} Find the point that maximises the expected improvement. The library\linebreak \texttt{scipy.optimize} could be useful.

\paragraph{Q4.} Run the experiment and update your model

\paragraph{Q5.} Repeat Questions 3 and 4. Do not forget to indicate in your report the best parameters you obtained and associated distance of the throw.

\end{document}
